%============================================================================
% tento soubor pouzijte jako zaklad
% (c) 2008 Michal Bidlo
% E-mail: bidlom AT fit vutbr cz
%============================================================================
% kodovaní: UTF-8 (zmena prikazem iconv, recode nebo cstocs)
%----------------------------------------------------------------------------
% zpracování: make, make pdf, make desky, make clean
%============================================================================
% Šablonu upravil: Ing. Jaroslav Dytrych, idytrych@fit.vutbr.cz
%============================================================================
\documentclass[english, zadani]{fitthesis} % bez zadání - pro začátek práce, aby nebyl problém s překladem
%\documentclass[zadani]{fitthesis} % odevzdani do wisu - odkazy jsou barevné
%\documentclass[zadani,print]{fitthesis} % pro tisk - odkazy jsou černé
%\documentclass[english,print]{fitthesis} % pro tisk - odkazy jsou černé
% * Je-li prace psana v anglickem jazyce, je zapotrebi u tridy pouzit 
%   parametr english nasledovne:
%      \documentclass[english]{fitthesis}
% * Je-li prace psana ve slovenskem jazyce, je zapotrebi u tridy pouzit 
%   parametr slovak nasledovne:
%      \documentclass[slovak]{fitthesis}

\usepackage[czech,slovak,english]{babel}
\usepackage[utf8]{inputenc} %kodovani
\usepackage[T1]{fontenc}
\usepackage{cmap}
\usepackage{url}
\DeclareUrlCommand\url{\def\UrlLeft{<}\def\UrlRight{>} \urlstyle{tt}}

%zde muzeme vlozit vlastni balicky
\usepackage{listings}
\usepackage[toc,page,header]{appendix}
\RequirePackage{titletoc}
\ifczech
  \usepackage{ae}
\fi

\usepackage[shadow]{todonotes}
%\usepackage[shadow,disable]{todonotes}
\setlength{\marginparwidth}{2.9cm}
\reversemarginpar
\newcommand{\todof}[1]{\todo[fancyline]{#1}}

%cislovani obrazku
\renewcommand{\thefigure}{\arabic{figure}}
\numberwithin{figure}{chapter}

%rovnice
\usepackage{amsmath}
\renewcommand{\theequation}{\arabic{equation}}
\numberwithin{equation}{chapter}


%tabulky
\usepackage{longtable}
\renewcommand{\thetable}{\arabic{table}}
\numberwithin{table}{chapter}
\usepackage{array}

%font na zdrojaky
\usepackage{courier}

%prostredi pro samplovani
\usepackage{listings}
\lstset{
	basicstyle=\small\ttfamily,
	columns=flexible,
	breaklines=true,
	aboveskip=0pt,
	belowskip=0pt
}

\definecolor{punct}{RGB}{200,40,40}
\definecolor{string}{RGB}{15,70,150}
\lstdefinelanguage{json}{
	basicstyle=\normalfont\ttfamily,
	numbers=left,
	numberstyle=\scriptsize,
	stepnumber=1,
	numbersep=8pt,
	showstringspaces=false,
	breaklines=true,
	frame=single,
	tabsize=9,
	stringstyle=\color{string},
	morestring=[b]",
	literate=
	{:}{{{\color{punct}{:}}}}{1}
	{,}{{{\color{punct}{,}}}}{1}
	{\{}{{{\color{punct}{\{}}}}{1}
	{\}}{{{\color{punct}{\}}}}}{1}
	{[}{{{\color{punct}{[}}}}{1}
	{]}{{{\color{punct}{]}}}}{1}
}


%seznam zkratek
\usepackage[toc,nonumberlist,nopostdot,translate=false]{glossaries} %translate kvuli konfliktu s babelem
\makeglossaries
%novy soubor pro seznam zkratek a jejich popis
\newglossaryentry{rnn}{
	name=RNN,
	description={Recurrent neural network.}
}

\newglossaryentry{cnn}{
	name=CNN,
	description={Convolutional neural network.}
}

\newglossaryentry{lstm}{
	name=LSTM,
	description={Long Short-Term Memory -- type of layer in recurrent nets.}
}

\newglossaryentry{gru}{
	name=GRU,
	description={Gated Recurrent Unit -- type of layer in recurrent nets.}
}

\newglossaryentry{relu}{
	name=ReLU,
	description={Rectified Linear Unit -- type of layer in convolutional nets.}
}

\newglossaryentry{coco}{
	name=MS COCO,
	description={Microsoft Common Objects in Context -- computer vision dataset and popular challenge.}
}

\newglossaryentry{mert}{
	name=MERT,
	description={Minimum Error Rate Training.}
}

\newglossaryentry{sgd}{
	name=SGD,
	description={Stochastic Gradient Descent.}
}

\newglossaryentry{bleu}{
	name=BLEU,
	description={Bilingual Evaluation Understudy.}
}

\newglossaryentry{meteor}{
	name=METEOR,
	description={Metric for Evaluation of Translation with Explicit Ordering.}
}

\newglossaryentry{cider}{
	name=CIDEr,
	description={Consensus-based Image Description Evaluation.}
}

\newglossaryentry{vgg}{
	name=VGG,
	description={16-layer convolutional neural net from \cite{DBLP:journals/corr/SimonyanZ14a}.}
}

\newglossaryentry{dmsm}{
	name=DMSM,
	description={Deep Multimodal Similarity Model.}
}

%vysazeni hezkeho C++
\newcommand{\CC}{C\nolinebreak\hspace{-.05em}\raisebox{.4ex}{\tiny\bf +}\nolinebreak\hspace{-.05em}\raisebox{.4ex}{\tiny\bf +}}


\input{pisma.tex}

% vypne funkci nové šablony, která automaticky nahrazuje uvozovky,
% aby nebyly prováděny nevhodné náhrady v popisech API apod.
\csdoublequotesoff

% =======================================================================
% balíček "hyperref" vytváří klikací odkazy v pdf, pokud tedy použijeme pdflatex
% problém je, že balíček hyperref musí být uveden jako poslední, takže nemůže
% být v šabloně
\ifWis
\ifx\pdfoutput\undefined % nejedeme pod pdflatexem
\else
  \usepackage{color}
  \usepackage[unicode,colorlinks,hyperindex,plainpages=false,pdftex]{hyperref}
  \definecolor{links}{rgb}{0.4,0.5,0}
  \definecolor{anchors}{rgb}{1,0,0}
  \def\AnchorColor{anchors}
  \def\LinkColor{links}
  \def\pdfBorderAttrs{/Border [0 0 0] }  % bez okrajů kolem odkazů
  \pdfcompresslevel=9
\fi
\else % pro tisk budou odkazy, na které se dá klikat, černé
\ifx\pdfoutput\undefined % nejedeme pod pdflatexem
\else
  \usepackage{color}
  \usepackage[unicode,colorlinks,hyperindex,plainpages=false,pdftex,urlcolor=black,linkcolor=black,citecolor=black]{hyperref}
  \definecolor{links}{rgb}{0,0,0}
  \definecolor{anchors}{rgb}{0,0,0}
  \def\AnchorColor{anchors}
  \def\LinkColor{links}
  \def\pdfBorderAttrs{/Border [0 0 0] } % bez okrajů kolem odkazů
  \pdfcompresslevel=9
\fi
\fi

%Informace o praci/projektu
%---------------------------------------------------------------------------
\projectinfo{
  %Prace
  project=DP,            %typ prace BP/SP/DP/DR
  year=2016,             %rok
  date=\today,           %datum odevzdani
  %Nazev prace
  title.cs={Popis fotografií pomocí rekurentních neuronových sítí},  %nazev prace v cestine ci slovenstine
  title.en={Image Captioning with Recurrent Neural Networks}, %nazev prace v anglictine
  %Autor
  author={Jakub Kvita},   %cele jmeno a prijmeni autora
  author.name={Jakub},   %jmeno autora (pro citaci)
  author.surname={Kvita},   %prijmeni autora (pro citaci)
  author.title.p=Bc., %titul pred jmenem (nepovinne)
  %author.title.a=PhD, %titul za jmenem (nepovinne)
  %Ustav
  department=UPGM, % doplnte prislusnou zkratku dle ustavu na zadani: UPSY/UIFS/UITS/UPGM
  %Skolitel
  supervisor=Michal Hradiš, %cele jmeno a prijmeni skolitele
  supervisor.name={Michal},   %jmeno skolitele (pro citaci)
  supervisor.surname={Hradiš},   %prijmeni skolitele (pro citaci)
  supervisor.title.p=Ing.,   %titul pred jmenem (nepovinne)
  supervisor.title.a={Ph.D.},    %titul za jmenem (nepovinne)
  %Klicova slova, abstrakty, prohlaseni a podekovani je mozne definovat 
  %bud pomoci nasledujicich parametru nebo pomoci vyhrazenych maker (viz dale)
  %===========================================================================
  %Klicova slova
  %keywords.cs={Klíčová slova v českém jazyce.}, %klicova slova v ceskem ci slovenskem jazyce
  %keywords.en={Klíčová slova v anglickém jazyce.}, %klicova slova v anglickem jazyce
  %Abstract
  %abstract.cs={Výtah (abstrakt) práce v českém jazyce.}, % abstrakt v ceskem ci slovenskem jazyce
  %abstract.en={Výtah (abstrakt) práce v anglickém jazyce.}, % abstrakt v anglickem jazyce
  %Prohlaseni
  %declaration={Prohlašuji, že jsem tuto bakalářskou práci vypracoval samostatně pod vedením pana ...},
  %Podekovani (nepovinne)
  %acknowledgment={Zde je možné uvést poděkování vedoucímu práce a těm, kteří poskytli odbornou pomoc.} % nepovinne
}

%Abstrakt (cesky, slovensky ci anglicky)
\abstract[cs]{Tato práce se zabývá automatickým generovaním popisů obrázků s využitím několika druhů neuronových sítí. Práce je založena na článcích z MS COCO Captioning Challenge 2015 a znakových jazykových modelech, popularizovaných A. Karpathym. Navržený model je kombinací konvoluční a rekurentní neuronové sítě s architekturou kodér--dekodér. Vektor reprezentující zakódovaný obrázek je předáván jazykovému modelu jako hodnoty paměti LSTM vrstev v síti. Práce zkoumá, na jaké úrovni je model s takto jednoduchou architekturou schopen popisovat obrázky a jak si stojí v porovnání s ostatními současnými modely. Jedním ze závěrů práce je, že navržená architektura není dostatečná pro jakýkoli popis obrázků.}

\abstract[en]{In this work I deal with automatic generation of image captions by using multiple types of neural networks. Thesis is based on the papers from MS COCO Captioning Challenge 2015 and character language models, popularized by A. Karpathy. Proposed model is combination of convolutional and recurrent neural network with encoder--decoder architecture. Vector representing encoded image is passed to language model as memory values of LSTM layers in the network. This work investigate, whether model with such simple architecture is able to generate captions and how good it is in comparison to other contemporary solutions. One of the results is that the proposed architecture is not sufficient for any image captioning task.}

%Klicova slova (cesky, slovensky ci anglicky)
\keywords[cs]{rekurentní neuronové sítě, RNN, konvoluční neuronové sítě, CNN, popisování obrázků, LSTM, GRU, MS COCO, Torch, hluboké učení}
\keywords[en]{recurrent neural networks, RNN, convolutional neural networks, CNN, image captioning, LSTM, GRU, MS COCO, Torch, deep learning}

%Prohlaseni (u anglicky psane prace anglicky, u slovensky psane prace slovensky)
\declaration{I hereby certify that this thesis is a presentation of my original research work and I have excercised reasonable care to ensure it does not to the best of my knowledge breach any law of copyright. Wherever contributions of others are involved, every effort is made to indicate this clearly, with due reference to the literature, and acknowledgement of collaborative research and discussions. The work was done under the guidance of Michal Hradiš at the Brno University of Technology.}

%Podekovani (nepovinne, nejlepe v jazyce prace)
\acknowledgment{Access to computing and storage facilities owned by parties and projects contributing to the National Grid Infrastructure MetaCentrum, provided under the programme \textquotedblleft Projects of Large Research, Development, and Innovations Infrastructures\textquotedblright (CESNET LM2015042), is greatly appreciated.}

\begin{document}
  % Vysazeni titulnich stran
  % ----------------------------------------------
  \maketitle
  % Obsah
  % ----------------------------------------------
  \tableofcontents
  
  % Seznam obrazku a tabulek (pokud prace obsahuje velke mnozstvi obrazku, tak se to hodi)
\ifczech
  \renewcommand\listfigurename{Seznam obrázků}
\fi
\ifslovak
  \renewcommand\listfigurename{Zoznam obrázkov}
\fi

   \listoffigures
   
\ifczech
  \renewcommand\listtablename{Seznam tabulek}
\fi
\ifslovak
  \renewcommand\listtablename{Zoznam tabuliek}
\fi

  % \listoftables 

  % Text prace
  % ----------------------------------------------
  %=========================================================================

\chapter{Introduction}
\todo[inline]{Klasicky popis toho co se tady bude dit, jak je to dulezite, atd.}

\chapter{Neural networks} 
General idea of neural networks was slowly emerging after World War II. Perceptron, as a single neuron unit, was created in 1958 by Frank Rosenblatt\footnote{The perceptron: A probabilistic model for information storage and organization in the brain. Rosenblatt, F. Psychological Review, Vol 65(6), Nov 1958, 386--408.}, but became popular only after creation of backpropagation algorithm in 1975. At that time neural nets have not reached massive popularity, not because they are not working, but due to small computing power of machines back then and lack of datasets. Recently (after 2000) neural nets became popular again, rebranded as \textquoteleft Deep Learning\textquoteright, because researchers realized that it is possible and very useful to stack neural nets on top of each other and create deep architectures, which are more practical than shallow ones. During this reinvention neural nets have been successfully applied in multiple fields like computer vision, speech recognition and natural language processing.

Since then various useful architectures and algorithms are now introduced almost every month. There is vast amount of various architectures and algorithms, in this chapter, I will describe only a couple -- those used in this thesis.

	\section{Recurrent neural nets}
Feedforward neural nets are extremely powerful models, which can be highly parallelized. Despite that, they can be only applied to problems with inputs and outputs, which have fixed dimensionality (e.g. one-hot encoding vectors). This is a serious drawback, as many of the real-world problems are defined as sequences with lengths that are unknown to us in beforehand. Soon recurrent neural networks were introduced and they proved to be very useful to this kind of task. There is vast amount of recurrent neural networks, many not suitable for sequential tasks like Hopfield network, which are very successful in specific tasks, but nevertheless not useful for us now.

Apart from classification, which can be more precise when using sequences, one of the most important tasks is next value prediction. This core task can be then extended very simply to predict arbitrary number of future values. Prediction problems are all around us, from the weather forecast and stock market prediction to the autocomplete in smartphones or web browsers.

\begin{figure}[!ht]
	\centering
	\includegraphics[width=0.9\textwidth]{fig/RNN-unrolled.png}
	\caption{Unrolling of the recurrent neural net. (C. Olah 2015 \cite{colah-lstm})
		\label{fig:rnn-unroll}}
\end{figure}
%reference http://colah.github.io/posts/2015-08-Understanding-LSTMs/

We can understand recurrent neural networks as very deep forward nets with shared weights. It is called RNN unrolling and it is described in figure \ref{fig:rnn-unroll}. Layers of this very deep net spread in time, together with the input sequence. This is very innovative idea, which enabled training RNN with backpropagation through time. It also shows that, as very deep networks, they have vanishing or exploding gradient problem, which means that the network is not able to learn long-term dependencies, even though in theory it should. This is a serious issue, which is caused by iterating many times over the weights and the activation function with derivatives $>1$ (exploding gradient) or $<1$ (vanishing gradient). Gradient then dies out and learning stops for distant dependencies. Among others this problem has been solved by the LSTM unit described in part \ref{subsec:lstm}, which is most popular now and following research resulting in GRU described in part \ref{subsec:gru}.

		\subsection{LSTM -- Long Short-Term Memory}\label{subsec:lstm}

Long Short-Term Memory nets are special kind of recurrent network, capable of learning long-term dependencies. This architecture was introduced by Hochreiter \& Schmidhuber (1997) in \cite{Hochreiter:1997:LSM:1246443.1246450} after prior research of vanishing gradient problem. Later architecture was refined and popularized by other researchers and nowadays LSTM is most used and popular RNN architecture used.

The LSTM unit designed that it can remember a value for an arbitrary length of time. It contains gates that determine when the input is significant enough to remember, when it should keep or forget the value, and when it should output the value. To understand the flow of data, see the diagram of a simplified LSTM unit is on the figure \ref{fig:lstm}.

%\begin{figure}[!ht]
%	\centering
%	\includegraphics[width=0.7\textwidth]{./fig/LSTM-equations.png}
%	\caption{LSTM unit for equations
%		\label{fig:lstm-equations}}
%\end{figure}
\begin{figure}[!ht]
	\centering
	\includegraphics[width=0.7\textwidth]{./fig/LSTM3-var-peepholes.png}
	\caption{Variation of the LSTM unit. (C. Olah 2015 \cite{colah-lstm})
		\label{fig:lstm}}
\end{figure}
%reference http://colah.github.io/posts/2015-08-Understanding-LSTMs/

All these gates can be described by series of equations \eqref{eq:lstm1}$ \rightarrow $\eqref{eq:lstm6}. In each time slice the unit is using current input $ x_t $, last stored value $ c_{t-1} $ and unit output $ h_{t-1} $ to compute next state $ c_t $ and output $ h_t $. Variables $ i_t $, $ f_t $, $ o_t $ denotes value of input, forget and output gates which are used to control the information flow.

\belowdisplayskip=24pt
\begin{align}
	i_t \hspace{7pt}&=\hspace{7pt} \sigma(W_{xi}x_t + W_{hi}h_{t-1} + W_{ci}c_{t-1} + b_i) \label{eq:lstm1}\\
	f_t \hspace{7pt}&=\hspace{7pt} \sigma(W_{xf}x_t + W_{hf}h_{t-1} + W_{cf}c_{t-1} + b_f) \label{eq:lstm2}\\
	z_t \hspace{7pt}&=\hspace{7pt} \tanh(W_{xc}x_t + W_{hc}h_{t-1} + b_c) \label{eq:lstm3}\\
	c_t \hspace{7pt}&=\hspace{7pt} f_t\odot c_{t-1} + i_t\odot z_t \label{eq:lstm4}\\
	o_t \hspace{7pt}&=\hspace{7pt} \sigma(W_{xo}x_t + W_{ho}h_{t-1} + W_{co}c_t + b_o) \label{eq:lstm5}\\
	h_t \hspace{7pt}&=\hspace{7pt} o_t\odot \tanh(c_t) \label{eq:lstm6}\\[16pt]
	\sigma(x) \hspace{7pt}&=\hspace{7pt} \frac{1}{1+e^{-x}} \label{eq:lstm7}
\end{align}

LSTM based on these equations is using total of 11 weight matrices and 4 bias vectors for computations and sigmoid function $ \sigma $ defined in the equation \eqref{eq:lstm7} and the operation $ \odot $ denotes the element-wise vector product. Equations described in this work are not the only way how to create an LSTM unit, but they will be used later while implementing the proposed model. Some of the versions are omitting \textquoteleft peephole connections\textquoteright, which allows gates to look at stored value $ C_{t-1} $, $ C_t $ or include only some of them.

Training of the LSTM based network can be performed effectively by standard methods like stochastic gradient descend in the form of backpropagation through time. Major problem with vanishing gradients during training described earlier is not an issue as backpropagated error is fed back to each of the gates.


		\subsection{GRU -- Gated Recurrent Unit}\label{subsec:gru}
Gated Recurrent Unit is slightly more dramatic variation on the LSTM theme from 2014 paper \cite{DBLP:journals/corr/ChoMGBSB14}. It combines hidden state of the unit $ h_t $ with the saved value $ C_t $, merges input and forget gates into one update gate and removes peephole connections. These changes are simplifying standard LSTM models, but not at the expense of performance, and cause rapid growth in popularity. Diagram of the GRU unit is on the figure \ref{fig:gru}.

\begin{figure}[!ht]
	\centering
	\includegraphics[width=0.7\textwidth]{./fig/LSTM3-var-GRU.png}
	\caption{Variation of a GRU unit. (C. Olah 2015 \cite{colah-lstm})
		\label{fig:gru}}
\end{figure}
%reference http://colah.github.io/posts/2015-08-Understanding-LSTMs/

\begin{align}
r_t \hspace{7pt}&=\hspace{7pt} \sigma(W_{xr}x_t + W_{hr}h_{t-1} + b_r) \label{eq:gru1}\\
z_t \hspace{7pt}&=\hspace{7pt} \sigma(W_{xz}x_t + W_{hz}h_{t-1} + b_z) \label{eq:gru2}\\
\widetilde{h}_t \hspace{7pt}&=\hspace{7pt} \tanh(W_{xh}x_t + W_{hh}(h_{t-1}\odot r_t) + b_h) \label{eq:gru3}\\
h_t \hspace{7pt}&=\hspace{7pt} (1-z_t)\odot \widetilde{h}_t + z_t\odot h_{t-1} \label{eq:gru4}\\[16pt]
\sigma(x) \hspace{7pt}&=\hspace{7pt} \frac{1}{1+e^{-x}} \label{eq:gru5}
\end{align}

Equations \eqref{eq:gru1}$ \rightarrow $\eqref{eq:gru4} describe a version of GRU unit used in this work, with sigmoid function $ \sigma $ defined in equation \eqref{eq:gru5}. The operation $ \odot $ again denotes the element-wise vector product. While it is using only 4 weight matrices, 3 biases and just 1 state variable, researchers studied whether this can achieve at least same performance as previous LSTM unit.

Last year, study by Chung \cite{DBLP:journals/corr/ChungGCB14} was done, where different types of recurrent units were compared on the polyphonic music datasets. In this task LSTM and GRU were significantly better than all the other architectures, with GRU slightly in the lead. Generally, researchers agree that most of the LSTM variations, including GRU, are roughly on the same performance level. In \cite{DBLP:journals/corr/GreffSKSS15} GRU is an average variation, slightly better than vanilla LSTM, with much simpler architecture.

In paper \cite{DBLP:conf/icml/JozefowiczZS15}, which emphasized variety of tasks and the data, GRU outperformed LSTM unit on all tasks with the exception of language modeling. There are multiple approaches to model languages and in this work I will explore different type than the one mentioned in Jozefowicz's \cite{DBLP:conf/icml/JozefowiczZS15} paper. More will be explained in following chapters. Interestingly they also found that LSTM nearly matched the GRU's performance, when its forget gate bias was initialized to 1 and not to naive initialization around 0.
It is also worth mentioning that Jozefowicz in his paper discovered several architectures similar to GRU, but with slightly better general performance. They were found by evolutionary algorithm working on candidate architectures  represented by the computational graph.


		\subsection{Language modeling and word embeddings}

With the addition of LSTM units, recurrent neural nets quickly showed good performance in many different types of sequence processing like speech recognition from sound waves, signal prediction and language modeling. These result were further improved when researchers started stacking LSTMs on top of each other like pancakes.

Text is represented by discrete values and is usually presented to network in form of input vectors with one-hot encoding\footnote{One-hot encoded vector has exactly one high ('1') value and all the others low ('0').}. If we have a task with $ K $ classes, class $ i $ will be represented by a vector $ V $ of length $ K $. All the entries of $ V $ will be switched off to 0, except $ V_i $, which will have the value of 1. Vector $ V $ is simultaneously a degenerated multinomial probability distribution of the current input. If the output has the same shape as input, it can be simply created by softmax function at the output layer. Result will be proper multinomial distribution of next value, given current value.

%TODO pridat ukazky vektoru a pravdepodobnostniho rozlozeni, kdyz bude cas

At this point it is necessary to decide what will classes and defined vectors represent. In most cases, text prediction is performed at the word level. $ K $ is hence the number of words in the dictionary. This can cause some problems, as in bigger tasks dictionary often exceeds 100 000 records. This many classes require huge amount of training data to properly cover all the cases and high computational cost of the softmax layer is also an issue. This text representation cannot be used for texts not containing separate words, like multi-digit numbers. Nevertheless, state-of-the-art models have been using word-level representation. One of the advantages is no need to teach the net proper forms of the words. The net does not have to remember, how to spell the words properly and can learn other, more useful, features.

To solve the problem with extremely long input vectors, set of techniques called \emph{word embedding} were developed. They map words from the vocabulary to suitable vectors of real numbers in high dimensional space (around 50--1000 dimensions). Chosen vectors cannot be random, they are meaningful in order of performing some following task. For example Skip-gram model from \cite{DBLP:journals/corr/abs-1301-3781} mapped 783 millions words to vectors of 300 real numbers, while creating reasonable relationships between them.

%TODO vice rozebrat kdyz bude cas a misto, klidne s ukazkama na celou stranku
%http://colah.github.io/posts/2014-07-NLP-RNNs-Representations/
%https://www.gavagai.se/blog/2015/09/30/a-brief-history-of-word-embeddings/

Character level modeling has been considered and used as an alternative to word-level, but so far had slightly worse performance. Regardless, it is still considered as an option, because it has much simpler representation of input and output. Consider roughly $ 45 $ characters in English text and over $ 50 000 $ words created from them. Character level network is also more suited for Czech or Russian and other fusional languages\footnote{Fusional language is a type of language distinguished by its tendency to overlay many morphemes to denote grammatical, syntactic, or semantic change.}, which heavily use prefixes and suffixes to create new words. This is also an ability, which cannot be overlooked, as it is not available for word level.
Character level models have usually smaller vocabulary size and have to be trained longer, as they need to learn spelling of the words on top of the same features of word level. With the properly trained character level model we can benefit from its much greater generative abilities, than we can achieve with word-level.

	\section{Convolutional neural nets}
\todo[inline]{Kratky uvod do toho, kde se pouzivaji, popis jak funguji.}
\todo[inline]{Neni potreba davat subsekce na vrstvy, staci popsat jak to funguje vsechno dohromady, jednotlive vrstvy ve vetach v jednom odstavci. Obrazek. V diplomce rozpracovat vic.}

\chapter{Experiments}
\todo[inline, color=magenta!60]{Asi kapitola jen na semestralni projekt. V diplomce ji odstranim.}
\todo[inline]{Jak se to implementuje, jake knihovny se pouzivaji - Caffe, Theano, TensorFlow, Torch. Popsat ze Torch bude v tehle kapitole.}
\todo[inline]{Budu popisovat veci co jsem zkousel implementovat v Torchi.}

	\section{Torch}
\todo[inline, color=cyan!60]{Torch se zrecykluje do diplomky.}
\todo[inline]{Udelat tady tabulku o ruznych balicich co torch ma}
\todo[inline]{Jak funguji rekurentni site v Torchi.}
\todo[inline]{Nacitani modelu z Caffe, ukladani v Torchi...}

		\subsection{nn, nngraph}
		\todo[inline]{Linky na knihovny v poznamkach pod carou.}
		
		\subsection{rnn}
		
		\subsection{Other packages}
		\todo[inline]{loadcaffe, optim,...}

	\section{Predicting next character in sequence}
\todo[inline]{Implementace sekce Language modeling, jak se to konkretne dela.}
\todo[inline]{Jak jsem to udelal, co to dela, ukazky.}
\todo[inline]{Karpathyho char-rnn}
\cite{char-rnn}

\chapter{Image caption generation}
\todo[inline]{Znovu uvod k tomu jak je to dulezite a tentokrat jak na tom lidi pracuji, co je potreba a jak se to hodnoti.}

	\section{Related Work}
\todo[inline]{Dat tomu nejake lepsi jmeno, clanky o popisovani obrazku ktere jsem cetl, pouzil.}

		\subsection{Show and Tell: A Neural Image Caption Generator}
		\todo[inline]{Clanek z Coco od Googlu.}
		\cite{DBLP:journals/corr/VinyalsTBE14}
		
		\todo[inline]{Zminit i strojovy preklad (Sequence to Sequence Learning with Neural Networks), architektura encoder, decoder}
		\cite{DBLP:journals/corr/SutskeverVL14}
		
		\subsection{Show, Attend and Tell: Neural Image Caption Generation with Visual Attention}
		\todo[inline]{Clanek z Coco z Montrealu/Toronta}
		\cite{DBLP:journals/corr/XuBKCCSZB15}
		
		\subsection{From Captions to Visual Concepts and Back}
		\todo[inline]{Clanek z Coco od Microsoftu, mrknout se i na pokracovani v druhem clanku}
		\cite{DBLP:journals/corr/FangGISDDGHMPZZ14}
		
		\subsection{Long-term Recurrent Convolutional Networks for Visual Recognition and Description}
		\todo[inline]{Clanek z Coco z berkeley}
		\cite{DBLP:journals/corr/DonahueHGRVSD14}


	\section{Datasets}
\label{sec:datasets}		
Big datasets are necessary requirement in training recurrent neural nets, together with sufficient computing power. As access to machines and hardware suitable for training has been made extremely easy, obtaining enough data become the biggest problem. All the descriptions in the image captioning datasets have to be human generated, which is very expensive. This is one of the reasons, not many specialized datasets are created.

There are two main options how to get images and captions. First, using user-generated data from an online service, most commonly Flicker. However, captions are not made specifically for the task and could be prone to error. Second option is to create captions directly for use in the dataset. Amazon Mechanical Turk\footnote{Amazon Mechanical Turk is crowdsourced Internet marketplace to perform tasks that computers are currently unable to do.} is heavily used for this task. All datasets mentioned here are created this way.

Flickr8k \cite{dataset-flickr8k} was one of the first datasets created for this purpose. It has been later expanded into Flickr30k \cite{dataset-flickr30k}. MS COCO \cite{DBLP:journals/corr/ChenFLVGDZ15} is dataset created by Microsoft for their captioning challenge. CIDEr \cite{Vedantam_2015_CVPR} datasets PASCAL-50S, ABSTRACT-50S are youngest mentioned, designed specifically for evaluation with the CIDEr metric.

\def\arraystretch{1.2}%  1 is the default, change whatever you need
%\setlength\extrarowheight{20pt}%add points to the row height

\begin{center}
	\begin{longtable}{|l|m{2cm}|m{2cm}|m{4.7cm}|}			
		\caption{Image captioning datasets.} \label{tab:datasets} \\

	   	\hline
	   	\multicolumn{1}{|l|}{\textbf{Name}} & 
	   	\textbf{Images} & 
	   	\textbf{Captions per image} & 
	   	\textbf{Note} \\
	   	\hline \hline
		\endhead
		
	   	Flickr8k\footnote{Flickr8k project page: \url{http://nlp.cs.illinois.edu/HockenmaierGroup/8k-pictures.html}} &
		   	\multicolumn{1}{r|}{8 092} &
		   	\multicolumn{1}{c|}{5} &
		   	Focused on people or animals (mainly dogs) performing some specific action. \\ \hline
	   	Flickr30k\footnote{Flickr30k project page: \url{http://shannon.cs.illinois.edu/DenotationGraph/}} & 
		   	\multicolumn{1}{r|}{31 783} &
		   	\multicolumn{1}{c|}{5-6} &
		   	An extension of Flickr8k dataset. \\ \hline
	   	MS COCO\footnote{MS COCO project page: \url{http://mscoco.org/dataset/}} &
		   	\multicolumn{1}{r|}{120 000} &
		   	\multicolumn{1}{c|}{5} &
		   	Images are divided - 80 000 for training and 40 000 for testing purposes. \\ \hline
	   	PASCAL-50S\footnote{\label{ft:cider}PASCAL-50S and ABSTRACT-50S page: \url{http://ramakrishnavedantam928.github.io/cider/}} &
		   	\multicolumn{1}{r|}{1 000} &
		   	\multicolumn{1}{c|}{50} &
		   	Built upon images from the UIUC Pascal Sentence Dataset. \\ \hline
	   	ABSTRACT-50S\footnote{See footnote \ref{ft:cider}.} &
		   	\multicolumn{1}{r|}{500} &
		   	\multicolumn{1}{c|}{50} &
		   	Built upon images from the Abstract Scenes Dataset. No photos.\\ \hline
	\end{longtable}
\end{center}

	\section{Evaluation}
	
Recent progress in fields like machine translation, which are very similar to image captioning, caused spike of interest in evaluating regular text output accuracy. Although it is sometimes not clear if a description of an image is best option available, some degree of assessment is possible. The best results can be obtained by asking live raters to give a score on the usefulness of each description. Subjective scores can vary, but it can be averaged by giving same description to multiple raters. However this method consumes tremendous amount of time and usually external raters are necessary. Tools like Amazon Mechanical Turk can be used to great extent, but need for automated tools is evident.

		\subsection{Automated metrics}
		
Assuming that one has access to human generated captions, which is ground truth in our case, completely automated metrics exists. Even though all of them compute how alike are generated to human descriptions, different approaches are used. One metric can use several different settings with slight changes in the algorithm. This raises the question, how can we compare results of different works, despite using the \textquoteleft same\textquoteright \ evaluation method. Microsoft group of researchers addresses this issue in \cite{DBLP:journals/corr/ChenFLVGDZ15}. They created an evaluation server\footnote{MS COCO evaluation server available through \url{http://mscoco.org/dataset/\#captions-upload}.} which have many automated metrics, with several configurations, including all mentioned here. It will serve as a reference point for comparing image captioning models.
		
The most commonly used metric has been BLEU (Bilingual Evaluation Understudy) \cite{Papineni:2002:BMA:1073083.1073135}, which was created in 2002 to evaluate quality of machine translated text from one language to another. Scores are computed on individual segments, usually sentences. BLEU has high correlation with human judgments and is still highly popular even for captioning tasks. However, it is becoming outdated as automatic methods are now outperforming humans. Four different variations of BLEU are used in MS COCO evaluation server.

METEOR (Metric for Evaluation of Translation with Explicit ORdering) \cite{Lavie:2007:MAM:1626355.1626389} is another metric for the evaluation of machine translation from 2007. It was designed to fix some problems of the BLEU metric and it can also look for synonyms and perform stemming on input words.
		
Last year, metric designed directly to caption evaluation called CIDEr (Consensus-based Image Description Evaluation) \cite{Vedantam_2015_CVPR} was introduced. This is still new metric, but with growing popularity as it correlate well with human judgment. Main idea of this metric is that given enough captions for the same image, metrics perform better. This can be seen in datasets introduced with it (see part \ref{sec:datasets}).		

\chapter{Model}
\todo[inline, color=cyan!60]{Do semestralniho projektu nebo az na diplomku?}
\todo[inline]{Design modelu, co chci pouzit, jake metody chci zkusit.}
\todo[inline]{Polozit si principialni otazku a zjistit jestli to nejak pomuze, jak to funguje.}

	\section{Architecture}
\todo[inline]{Architektura modelu, jake matematicke modely jsem pouzil, bez implementacnich detailu.}

	\section{Training details}
\todo[inline]{Popis pomoci jakeho algoritmu jsme trenovali, s jakyma parametrama, minibatches, datasety.}

%\chapter{Implementation}
%\todo[inline, color=cyan!60]{Bude az v diplomce, ne semestralnim projektu}
%
%	\section{Torch framework}
%\todo[inline]{Popis Torche, ruznych druhu modulu ktere ma, ktere jsem pouzil ja.}
%
%	\section{Parallelization}
%\todo[inline]{Nejaky popis toho ze to lze trenovat na gpu, jak se to dela v Torchi..}	
%
%\chapter{Results and model evaluation}
%\todo[inline, color=cyan!60]{Bude az v diplomce, ne semestralnim projektu}
%\todo[inline]{Performance, analysis,... spravne pojmenovat}
%\todo[inline]{Kde se trenovalo? Na jakych strojich? Jak to bylo rychle? Jak to bylo na cpu pomale?}
%
%	\section{Speed}
%\todo[inline]{Jak bylo trenovani rychle, }

\chapter{Conclusion}
\todo[inline]{Udelat jeden zaver pro semestralni projekt, pak ho prepsat pro diplomku.}
%=========================================================================
 % viz. obsah.tex

  % Pouzita literatura
  % ----------------------------------------------
\ifslovak
  \makeatletter
  \def\@openbib@code{\addcontentsline{toc}{chapter}{Literatúra}}
  \makeatother
  \bibliographystyle{czechiso}
\else
  \ifczech
    \makeatletter
    \def\@openbib@code{\addcontentsline{toc}{chapter}{Literatura}}
    \makeatother
    \bibliographystyle{czechiso}
  \else 
    \makeatletter
    \def\@openbib@code{\addcontentsline{toc}{chapter}{Bibliography}}
    \makeatother
    \bibliographystyle{plain}
  %  \bibliographystyle{alpha}
  \fi
\fi
%  \begin{flushleft}
  \bibliography{literatura} % viz. literatura.bib
%  \end{flushleft}

	% Seznam zkratek
	% ---------------------------------------------
\ifczech
\renewcommand\glossaryname{Seznam zkratek}
\else
\renewcommand\glossaryname{List of Abbreviations}
\fi

\newglossarystyle{mylong}{%
	\setglossarystyle{long}% base this style on the long style
	\renewenvironment{theglossary}%
	%                                     zkratka            popis           strany 
	{\begin{longtable}[l]{@{}>{\bfseries}p{0.2\textwidth}p{0.7\textwidth}p{0.1\textwidth}}}%
		{\end{longtable}}%
	\renewcommand*{\glsgroupskip}{}%mezery mezi skupinami
	\renewcommand{\arraystretch}{1.8}%mezery mezi zaznamy
}

\glossarystyle{mylong}
\printglossaries


  % Prilohy
  % ---------------------------------------------
  \appendix
\ifczech
  \renewcommand{\appendixpagename}{Přílohy}
  \renewcommand{\appendixtocname}{Přílohy}
  \renewcommand{\appendixname}{Příloha}
\fi
\ifslovak
  \renewcommand{\appendixpagename}{Prílohy}
  \renewcommand{\appendixtocname}{Prílohy}
  \renewcommand{\appendixname}{Príloha}
\fi
  \appendixpage

\ifslovak
  \section*{Zoznam príloh}
  \addcontentsline{toc}{section}{Zoznam príloh}
\else
  \ifczech
    \section*{Seznam příloh}
    \addcontentsline{toc}{section}{Seznam příloh}
  \else
    \section*{List of Appendices}
    \addcontentsline{toc}{section}{List of Appendices}
  \fi
\fi
  \startcontents[chapters]
  \printcontents[chapters]{l}{0}{\setcounter{tocdepth}{2}}
  \chapter{CD Content}

\todo[inline]{Predelat}

\begin{table}[h]
	\centering
	\renewcommand{\arraystretch}{1.2}
	\begin{tabular}{|l|l|}
		\hline
		Adresář & Obsah  \\
		\hline
		\hline
		LModeler/ & Projekt pro Netbeans bez knihoven. \\
		LModeler/src/ & Zdrojové soubory LModeleru v Javě. \\
		lib/ & Knihovny potřebné pro LModeler. \\
		runnable/ & Spustitelný JAR soubor s aplikací.\\
		javadoc/ & Vygenerovaná dokumentace. \\
		text/ & Zdrojové soubory této práce v LaTeXu. \\
		text/fig/ & Obrázky z této práce. \\
		\hline
	\end{tabular}
	\label{tab:cdcontent}
\end{table}

\chapter{\gls{coco} Annotation format} \label{chp:jsonAnnotation}
\begin{lstlisting}[language=json,firstnumber=1]
	{
		"info"		: info,
		"images"		: [image],
		"annotations"	: [annotation],
		"licenses"	: [license]
	}
	info {
		"year"		: int,
		"version"	: str,
		"description"	: str,
		"contributor"	: str,
		"url"		: str,
		"date_created"	: datetime
	}
	image {
		"id"		: int,
		"width"		: int,
		"height"		: int,
		"file_name"	: str,
		"license"	: int,
		"flickr_url"	: str,
		"coco_url"	: str,
		"date_captured"	: datetime
	}
	license {
		"id"		: int,
		"name"		: str,
		"url"		: str
	}
	annotation {
		"id"		: int,
		"image_id"	: int,
		"caption"	: str
	}
\end{lstlisting}

\chapter{Scripts Manuals}

\section{RNN pretraining}

\subsubsection{Training}

\begin{lstlisting}[firstnumber=1,breakindent=75pt]
~/captioning/pretrainRNN$ th pretraining.lua --help  
Usage: th [options] 

Train a RNN language model for generating image captions.

Options
-captionFile    JSON file with the input data (captions, image names). [~/COCO/captions_train2014.json]

-recurLayers    Number of recurrent layers. (At least one.) [3]
-hiddenUnits    Number of units in hidden layers. (At least one.) [300]
-dropout        Use dropout. [false]
-batchSize      Minibatch size. [15]
-printError     Print error once per N minibatches. [10]
-sample         Try to sample once per N minibatches. [100]
-saveModel      Save model once per N minibatches. [10000]
-modelName      Filename of the model and training data. [rnn.torch]
-modelDirectory Directory where to save the model. [~/RNN/]
\end{lstlisting}
\hspace{1cm}

\subsubsection{Sampling}

\begin{lstlisting}[firstnumber=1,breakindent=75pt]
~/captioning/pretrainRNN$ th sampling.lua --help
Usage: th [options] 

Sample a language model for generating image captions.

Options
-modelName Filename of the model and training data. [~/RNN/rnn.torch]
-N         How many captions will be generated. [4]
\end{lstlisting}
\hspace{1cm}

\section{Full model}

\subsubsection{Training}

\begin{lstlisting}[firstnumber=1,breakindent=75pt]
~/captioning$ th training.lua --help
Usage: th [options] 

Training of the CNN-RNN network for generating image captions.

Options
-captionFile    JSON file with the input data (captions, image names). [~/COCO/captions_train2014.json]
-imageDirectory Directory with the images named according to the caption file. [~/COCO/train2014/]

-pretrainedCNN  Path to a ImageNet pretrained CNN in Torch format. [~/CNN/VGG_ILSVRC_16_layers_fc7.torch]
-ft             Finetune CNN on the dataset. (Enable CNN training.) [false]

-pretrainedRNN  Path to a pretrained RNN. [~/RNN/2.0000__3x300.torch]

-rnnLayers      If no RNN is provided, number of recurrent layers while creating RNN. (At least one.) [3]
-rnnHidden      If no RNN is provided, number of units in hidden layers while creating RNN. (At least one.) [300]
-rnnDropout     If no RNN is provided, use dropout while creating RNN. [false]

-initLayers     How many reccurent layers initialize with CNN data. (0 - all of them) [0]
-batchSize      Minibatch size. [15]
-printError     Print error once per N minibatches. [10]
-sample         Try to sample once per N minibatches. [100]
-saveModel      Save model once per N minibatches. [10000]
-modelName      File name of the saved or loaded model and training data. [model.torch]
-modelDirectory Directory where to save the model. [~/combined_model/]
\end{lstlisting}
\hspace{1cm}

\subsubsection{Sampling}

\begin{lstlisting}[firstnumber=1,breakindent=75pt]
~/captioning$ th sampling.lua --help
Usage: th [options] 

Generate captions for images with trained model.

Options
-modelName Name of the model to be loaded. [model.torch]
-N         How many captions will be generated. [3]
\end{lstlisting}
\hspace{1cm}

\section{Bag of Words model}

\subsubsection{Training}

\begin{lstlisting}[firstnumber=1,breakindent=75pt]
~/captioning/bagOfWords$ th training.lua --help
Usage: th [options] 


\end{lstlisting}
\hspace{1cm}

\todo[inline]{Dodelat.}

%\chapter{RelaxNG Schéma konfiguračního soboru}
%\chapter{Plakat}

 % viz. prilohy.tex
\end{document}
